

 % !TEX encoding = UTF-8 Unicode

\documentclass[a4paper]{report}

\usepackage[utf8x,utf8]{inputenc} % make weird characters work
\usepackage[croatian]{babel}
%\usepackage[english,serbianc]{babel}
\usepackage{amssymb}

\usepackage{color}
\usepackage{url}
\usepackage[unicode]{hyperref}
\hypersetup{colorlinks,citecolor=green,filecolor=green,linkcolor=blue,urlcolor=blue}

\newcommand{\odgovor}[1]{\textcolor{blue}{#1}}

\begin{document}

\title{Osnove formalne semantike programskih jezika\\ \small{Isidora Đurđević, Ana Stanković, Milica Đurić}}
\date{1.~maj 2017.}
\maketitle

\tableofcontents

\chapter{Recenzent \odgovor{--- ocena:}}

\section{O čemu rad govori?}
% Напишете један кратак пасус у којим ћете својим речима препричати суштину рада (и тиме показати да сте рад пажљиво прочитали и разумели). Обим од 200 до 400 карактера.

{Rad upućuje čitaoca u različite načine formalizacije semantike programskih jezika, i to u: operacionu, denotacionu i aksiomatsku semantiku. Pre same razrade, rad govori o potrebi za formalizacijom semantike. Dodatno, za svaki formalizam, dat je po jedan primer funkcionisanja te semantike.}

\section{Krupne primedbe i sugestije}
% Напишете своја запажања и конструктивне идеје шта у раду недостаје и шта би требало да се промени-измени-дода-одузме да би рад био квалитетнији.

{U delu teksta o tranzicionim sistemima smatram da treba dodatno pojasniti vrste konfiguracija. Neki predlozi: šta znači pojam konfiguracije? U kom smislu će naredba $S$ biti izvršena ‚‚od stanja $s$''? Šta se tačno smatra konfiguracijom u datim dvama nabrajanjima? (Možda bi trebalo konfiguracije nekako posebno naznačiti, bilo podebljavanjem, iskošavanjem, promenom fonta ili nekako drugačije.)}\\
\odgovor{U radu je dodato objašnjenje pojma konfiguracije, takođe, posebno su definisane obe vrste operacione semantike gde se kroz primere može jasno razumeti značenje konfiguracije. Na osnovu ovih izmena trebalo bi da bude recezentu jasno zbog čega nije neophodno posebno naznačiti konfiguracije.}

{Rečeno je da definiciju za tranzicionu relaciju možemo videti u Tabeli 1, ali meni nije bilo jasno šta je tačno definicija toga u toj tabeli.}\\
\odgovor{Tranziciona relacija se definiše pravilima iz tabele, to jest pravilima za neki konkretan primer. Proširen je odeljak za prirodnu semantiku, kao i za strukturnu operacionu semantiku.}

{Predlog je da se nakon paragrafa ‚‚Pored dokazivanja korektnosti programa i algoritama, uloga aksiomatske sematike je i dokazivanje ispravnosti...'' dodaju reference ka knjigama ili radovima koji pokazuju navedene uloge.}  \\
\odgovor {Dodata je referenca ka knjizi.}
	
{Paragraf nakon Primera 5.1 je isti kao i zaključak koji sledi, pa je višak.} \\
\odgovor {Obrisan je paragraf koji je dupliran. }

\section{Sitne primedbe}
% Напишете своја запажања на тему штампарских-стилских-језичких грешки

{Ima dosta sitnijih grešaka, pogotovo u početnim sekcijama rada. Takođe, imam i nekoliko stilskih primedbi i nekih nedoumica, koje ću izložiti tim redom u nastavku.}

Greške:
\begin{itemize}
	\item višak zapete iza ‚‚u toku izvođenja'' \\
	\odgovor{Na mestu gde je bila zapeta stavljena je tačka zbog lepše strukture rečenica.} 
	\item ‚‚da d\^{a} potpunu'' umesto ‚‚da da potpunu''\\
	\odgovor{Ispravljeno u ‚‚da d\^{a} potpunu'' .}
	\item ‚‚naznačenoj literaturi'' umesto ‚‚naznačenim literaturama''\\
	\odgovor{Ispravljeno u ‚‚naznačenoj literaturi''.}
	\item izbaciti ‚‚u 1'' jer je jasno da je u uvodu već napomenuto (nema gde drugde)\\
	\odgovor{Obrisana je referenca na uvod.}
	\item ‚‚predstavlja'' umesto ‚‚predstavljaja''\\
	\odgovor{Ipravljeno u predstavlja.}
	\item ‚‚rezultati izvršavanja dobijeni'' umesto ‚‚rezultati izvršavanja dobijena''\\
	 \odgovor{Ispravljeno u ‚‚rezultati izvršavanja dobijeni''. }
	\item ‚‚u jeziku Pascal'' umesto ‚‚u Pascal jeziku''\\
	\odgovor{Ispravljeno u ‚‚u jeziku Pascal''.}
	\item ‚‚raspolaže instrukcijama'' umesto ‚‚raspolaže sa instrukcijama''\\
	 \odgovor{Ispravljeno u ‚‚raspolaže instrukcijama''.}
	\item znak \textit{$<$} umesto obrnutog znaka upitnika u instrukciji apstraktne mašine\\
	\odgovor{Ipravljeno u \textit{$>$}.}
	\item loša referenca nakon ‚‚Ona će imati dve vrste konfiguracija''\\
	\odgovor{Ispravljena referenca. Nije se poklapalo ime reference u .bib fajlu sa imenom u .tex fajlu.}
	\item ‚‚Pravilo sa praznim skupom'' umesto ‚‚Pravila sa praznim skupom''\\
	\odgovor{Ispravljeno u ‚‚Pravilo sa praznim skupom''.}
	\item loša referenca nakon ‚‚Pravila sa praznim skupom premisa se naziva aksiom''\\
	\odgovor{Ispravljena referenca. Nije se poklapalo ime reference u .bib fajlu sa imenom u .tex fajlu.}
	\item naziv tabele se stavlja iznad tabele (Tabela 1)\\
	\odgovor{Naziv tabele se sada nalazi iznad tabele.}
	\item ‚‚koje su potrebne'' umesto ‚‚koje su potreban''\\
	\odgovor{Ispravljeno u ‚‚koje su potrebne''.}
	\item u sekciji 5 ima dosta ogoljenih latiničnih slova \\
	\odgovor{Ispravljena su sva ogoljena latinična slova.}
	\item ‚‚Zasniva se na'' umesto ‚‚Zasniva na'' \\\\
	\odgovor{Ispravljeno u ‚‚Zasniva se na''.}
	\item ‚‚će biti određene'' umesto ‚‚će biti određen'' \\
	\odgovor{Ispravljeno u ‚‚će biti određene''. }
	\item ‚‚promenljiva n'' umesto ‚‚promenljive n'' \\
	\odgovor{Ispravljeno u  ‚‚promenljiva n''.}
	\item ‚‚odnosno'' umesto ‚‚odnoso'' \\
	\odgovor{Ispravljeno u ‚‚odnosno''. }
\end{itemize}

{Stilske primedbe:}
\begin{itemize}
	\item stavio bih sve izraze u {\tt $\backslash$verb|...| ili $\backslash$texttt\{...\}} okruženje umesto u {\tt \$...\$} okruženje radi čitljivosti i stavljanja do znanja da su u pitanju programski fragmenti\\
\odgovor{Primeri definisanja formalnih semantika su napisani u okviru {\tt $\backslash$texttt\{...\}} okruženja.}
	\item ‚‚nije povezana'' umesto ‚‚nema ništa''\\
	\odgovor{Ispravljeno u ‚‚nije povezana''.}
	\item u sekciji 2, na kraju prvog i na početku narednog paragrafa ponavlja se opažanje da sa porastom kompleksnosti koda raste nerazumevanje neformalnih definicija\\
	\odgovor{Obrisana je poslednja rečenica prvog pasusa, a dodata je nova u kojoj je se samo postavlja pitanje da li kompleksnost koda utiče na složenost neformalnih definicija.}
	\item predlog: zameniti ‚‚kompilator'' umesto ‚‚kompajler'' i ‚‚kompilacija'' ili ‚‚kompiliranje'' umesto ,,kompajliranje''\\
	 \odgovor{Svako pojavljivanje reči ‚‚kompajler'' je zamenjeno sa ‚kompilator''. Međutim, to nije bilo moguće uraditi na slici.}
	\item u delu o tranzicionim sistemima nedostaju termini na engleskom jeziku\\
\odgovor{Dodati engleski termini: transition system, transition relation, immediate constituents u delu o tranzicionim sistemima, poglavlje 2.1 i 2.1.1.}\\
\end{itemize}

{Nedoumice:}
\begin{itemize}
	\item čini mi se da je prestrogo reći da ‚‚\textit{ćemo} definisati realnu ili apstraktnu mašinu'' s obzirom da se odmah zatim govori kako nije pogodno uzeti realnu mašinu. Ako su autori želeli eksplicitno da kažu zašto nije podesno uzeti realnu mašinu, možda bi bilo lepše reći ‚‚Za svaki programski jezik možemo definisati realnu ili apstraktnu mašinu''\\
\odgovor{Rečenica \textit{‚‚Za svaki programski jezik ćemo definisati realnu ili apstraktnu mašinu, na kojoj ćemo pratiti promene stanja koje proizvode pojedini konstrukti programskog jezika''} zamenjena je rečenicom \textit{‚‚Za svaki programski jezik možemo definisati realnu ili apstraktnu mašinu,
na kojoj ćemo pratiti promene stanja koje proizvode pojedini konstrukti
programskog jezika.''} }\\
	\item u opštoj formi pravila pominje se $S_1'$, ali ne i kako ono učestvuje dalje u gornjem skupu pravila. Ovako izgleda kao da je višak. Možda treba dodati još jedan korak pre stavljanja tri tačke.\\
\odgovor{Tri tačke ovde ne predstavljaju preskočene korake, već preskočene premise. U paragrafu posle opšte forme pravile naznačeno je da se iznad linije nalazi skup premisa, a ispod linije se nalazi zaključak.}\\
	\item šta predstavlja $D$ u funkcijama značenja?\\
	\odgovor{Slučajno je zamenjeno $D$ sa $C$. Svako pojavljivanje $D$ je zamenjeno sa $C$.}
\end{itemize}

\section{Provera sadržajnosti i forme seminarskog rada}
% Oдговорите на следећа питања --- уз сваки одговор дати и образложење

\begin{enumerate}
\item Da li rad dobro odgovara na zadatu temu?\\
{Da, jer su pokrivene suštinski važne oblasti obuhvaćene temom.}
\item Da li je nešto važno propušteno?\\
{Nije. Eventualno je moglo da se nešto detaljnije napiše o tome koja semantika odgovara funkcionalnim/imperativnim jezicima.}
\odgovor{U poglavlju 2.1 i u poglavlju 2.2 su dodati pasusi koji govore da se za imperativne jezike koristi operaciona, a za funkcionalne jezike denotaciona semantika.}
\item Da li ima suštinskih grešaka i propusta?\\
{Nema, suštinski važne oblasti su dobro objašnjenje.}
\item Da li je naslov rada dobro izabran?\\
{Jeste. Odmah se može pretpostaviti o čemu će biti reč.}
\item Da li sažetak sadrži prave podatke o radu?\\
{Da.}
\item Da li je rad lak-težak za čitanje?\\
{Generalno je lak, ali na momente ima nejasnih stvari o kojima je bilo reči u prethodnom tekstu recenzije.}
\item Da li je za razumevanje teksta potrebno predznanje i u kolikoj meri?\\
{Zbog dobro napisanih uvodnih delova rada nije potrebno predznanje.}
\item Da li je u radu navedena odgovarajuća literatura?\\
{Da.}
\item Da li su u radu reference korektno navedene?\\
{Da.}
\item Da li je struktura rada adekvatna?\\
{Da.}
\item Da li rad sadrži sve elemente propisane uslovom seminarskog rada (slike, tabele, broj strana...)?\\
{Nedovoljan je broj referenci (treba 8, a ima 6). Ostalo je korektno.}\\
\odgovor{Dodate su neke reference i rad sada sadrži 10 referenci.}
\item Da li su slike i tabele funkcionalne i adekvatne?\\
{Da.}
\end{enumerate}

\section{Ocenite sebe}
% Napišite koliko ste upućeni u oblast koju recenzirate: 
% a) ekspert u datoj oblasti
% b) veoma upućeni u oblast
% c) srednje upućeni
% d) malo upućeni 
% e) skoro neupućeni
% f) potpuno neupućeni
% Obrazložite svoju odluku

{Upućenost u oblast: malo upućeni (d). Na nekoliko kurseva je bilo reči o ovoj temi, odnosno, o nekim njenim delovima.}

\chapter{Recenzent \odgovor{--- ocena:}}

\section{O čemu rad govori?}
% Напишете један кратак пасус у којим ћете својим речима препричати суштину рада (и тиме показати да сте рад пажљиво прочитали и разумели). Обим од 200 до 400 карактера.
Rad se bavi opisivanjem formalne semantike programskih jezika opisivanjem tri glavne vrste semantike. Pokazuju se nedostaci prirodnog jezika u opisivanju značenja programa, i sa druge strane, precizno definišu matematički modeli formalnih semantika.

\section{Krupne primedbe i sugestije}
% Напишете своја запажања и конструктивне идеје шта у раду недостаје и шта би требало да се промени-измени-дода-одузме да би рад био квалитетнији.
Neka objašnjenja su mogla biti malo opširnija u cilju  boljeg shvatanja materijala.
\odgovor{Recezent je trebalo da naglasi koja objašnjenja treba proširiti. Proširen je deo u kome se govori o operacionoj semantici.}

\section{Sitne primedbe}
% Напишете своја запажања на тему штампарских-стилских-језичких грешки
"...svrha joj je da opiše kako su ukupni rezultati izvršavanja dobijena." - dobijeni umesto dobijena. Svakako, promenio bih redosled reči u ovoj rečenici.\\
 \odgovor{Ispravljena je reč ,,dobijena'' u ,,dobijeni''. Redosled reči u rečenici jasno govori šta je zadatak prirodne semantike.}\\
Poslednja rečenica na strani 3 ima s umesto š u reči mašina.\\
\odgovor{Sva ogoljena latinična slova su ispravljena.}\\
 "Pravilo se sastoji od određenog broja premisi" - mislim da ovde treba da stoji premisa umesti premisi.\\
 \odgovor{Ispravljena je reč ,,premisi'' u reč ,,premisa''.}\\
  Greška u kucanju u reči "matematičkoj" (napisano "matematickoj") u odeljku 5. Ograničenja umesto ogranicenja u istom odeljku.\\
  \odgovor{Sva ogoljena latinična slova su ispravljena.}\\
   Rečenica "Tipično, programeri ih pišu kao komentari za funkcije..." trebalo bi da stoji komentare umesto komentari.\\
   \odgovor{Ispravljena je reč ,,komentari'' u reč ,,komentare''.}\\
   Poslednji pasus pre zaključka bi trebalo da pripada zaključku.\\
    \odgovor{Obrisan je dupliran pasus.}

\section{Provera sadržajnosti i forme seminarskog rada}
% Oдговорите на следећа питања --- уз сваки одговор дати и образложење

\begin{enumerate}
\item Da li rad dobro odgovara na zadatu temu?\\
Rad odgovara na zadatu temu kroz opisivanje tri vrste formalne semantike.
\item Da li je nešto važno propušteno?\\
S obzirom na moje (ne)znanje o ovoj oblasti, mislim da nije.
\item Da li ima suštinskih grešaka i propusta?\\
Nema.
\item Da li je naslov rada dobro izabran?\\
Naslov odgovara materijalu obrađenom u radu.
\item Da li sažetak sadrži prave podatke o radu?\\
Sažetak sadrži sve bitne stavke koje u daljem tekstu opisuje.
\item Da li je rad lak-težak za čitanje?\\
Rad je težak za čitanje zbog količine predznanja koje je potrebno da bi se tekst ispratio.
\item Da li je za razumevanje teksta potrebno predznanje i u kolikoj meri?\\
Potrebna je pozamašna količina znanja iz oblasti matematike i informatike.
\item Da li je u radu navedena odgovarajuća literatura?\\
Jeste.
\item Da li su u radu reference korektno navedene?\\
Jesu, sem u par slučajeva gde su ostale, valjda slučajno upitnici umesto referenci ([?]).
\item Da li je struktura rada adekvatna?\\
Ispoštovani su uslovi za izradu seminarskog rada.
\item Da li rad sadrži sve elemente propisane uslovom seminarskog rada (slike, tabele, broj strana...)?\\
Da.
\item Da li su slike i tabele funkcionalne i adekvatne?\\
Jesu.
\end{enumerate}

\section{Ocenite sebe}
% Napišite koliko ste upućeni u oblast koju recenzirate: 
% a) ekspert u datoj oblasti
% b) veoma upućeni u oblast
% c) srednje upućeni
% d) malo upućeni 
% e) skoro neupućeni
% f) potpuno neupućeni
% Obrazložite svoju odluku
Potpuno sam neupućen u temu opisanu ovim radom.


\chapter{Recenzent \odgovor{--- ocena:} }


\section{O cemu rad govori?}

Centralna tema ovog rada je  opisivanje formalne sematnike.Ukratko je opisano šta je sintaksa, a šta semantika programskih jezika, kao i njihivo znacenje.Navedeno je koji nacini zadavanja semantike postoje.Opisana je ukratko i formalna i neformalna semantika, kao i prednosti formalne u odnosu na neformalnu. Opisano je koje formalne semantike postoje i ukratko je objašnjena svaka od njih. 


\section{Krupne primedbe i sugestije}

U samom uvodu je, na nacn razumljiv i nekome kome ova tema nije familijarna, opisano šta je sintaksa, a šta semantika programskih jezika. Lepo je opisana i razlika izmedju formalnog i neformalnog zadavanja semantike. Možda bi bilo dobro pronaci dobar primer i na njemu prikazati uporedo formalnu i neformalnu semantiku.\\
\odgovor{U radu je na početku prikazan primer neformalne semantike uz upotrebu while naredbe, a uporedo prikazivanje formalne semantike na tom mestu nije moguće jer još uvek nisu definisani načini definisanja formalne semantike. Ideja je bila da se kroz različite primere prikažu tri vrste formalne semantike, dok je neformalna semantika poprilično intuitivna za jednostavne primere, pa bi bilo višak prikazivati je kod primera koji su dati za ilustraciju formalne semantike.}\\
 Na više mesta u radu se pominje koje je znacenje sintakse a koje semantike u programskim jezicima, što mislim da treba ostaviti samo na jednom mestu, na pocetku rada. \\
\odgovor{Samo u uvodu se pominje značenje sintakse i semantike, iz zaključka je obrisano ponovljeno definisanje.}\\
Dupliran je tekst na kraju rada, pasus pre zakljucka i zakljucak. \\
\odgovor{Obrisan je dupliran pasus.}\\
Prilikom opisivianja operacione semantike pri dnu strane cetiri, pominje se da ce biti opisane obe vrste operacione semantike, a opisana je samo prirodna  semantika.\\
\odgovor{Dodato je objašnjenje za strukturnu semantiku (odeljak 2.1.2).}\\
 Prikazana je tablica prirodne semantike za while petlju, možda je potrebno više taj primer približiti citaocu(npr. opis recima samog primera, kao što je uradjeno u delu vezanom za aksiomatsku semantiku primer Primer 4.1).    
\\
\odgovor{Dodat je poseban odeljak za prirodnu i strukturnu operacionu semantiku gde se u oba slučaja posmatra i objašnjava primer semantike za while petlju.}\\
\section{Sitne primedbe}

strana 3....kako su ukupni rezultati izvrsavanja dobijena. Nije ispravno, treba dobijeni.\\
\odgovor{Ispravljeno je u ,,dobijeni''.}\\
strana 4....[?] sta znaci ovakva referenca citaocu, greska ili je nepoznata referenca\\
\odgovor{Ispravljene su greške sa referencama u celom radu.}\\


\section{Provera sadržajnosti i forme seminarskog rada}


\begin{enumerate}
\item Da li rad dobro odgovara na zadatu temu?\\
Da.Posle citanja rada, citalac je dobro upoznat sa temom.
\item Da li je nešto važno propušteno?\\
Da. Nije odgovoreno na pitanja koja semantika najvise odgovara funckionalnim, a koja imperativnim programskih jezicima.\\
\odgovor{U poglavlju 2.1 i u poglavlju 2.2 su dodati pasusi koji govore da se za imperativne jezike koristi operaciona, a za funkcionalne jezike denotaciona semantika.}
\item Da li ima suštinskih grešaka i propusta?\\
Ne. U radu je tema dobro obradjena i na prilicno jednostavan nacin približena citaocu.
\item Da li je naslov rada dobro izabran?\\
Da. Naslov je suština celog teksta.
\item Da li sažetak sadrži prave podatke o radu?\\
Da. U sažetku je napisano šta ce biti tema i sve je dalje u tekstu pokriveno.
\item Da li je rad lak-težak za citanje?\\
Rad je prilicno lak za citanje i nekome ko nema mnogo strucnog znanja o samoj temi. Jedini deo koji je malo teži za razumevanje je deo vezan za operacionu semantiku.
\item Da li je za razumevanje teksta potrebno predznanje i u kolikoj meri?\\
Potrebno je osnovno poznavanje strukture programskih jezika.
\item Da li je u radu navedena odgovarajuca literatura?\\
Da.
\item Da li su u radu reference korektno navedene?\\
Da. Osim sto postoji [?] referenca. 
\item Da li je struktura rada adekvatna?\\
Da. Mada bi mozda trebalo struktuirati tako da se i po strukturi rada vidi da su operaciona, denotaciona i aksiomatska semantika podceline (vrste) formalne semantike.\\
\odgovor{Preuređene su sekcije, sada operaciona, denotaciona i aksiomatska semantika predstavlaju podsekcije formalne semantike.}
\item Da li rad sadrži sve elemente propisane uslovom seminarskog rada (slike, tabele, broj strana...)?\\
Uglavnom da. Jedino je broj referenci manji od propisanog za broj clanova tima. Za tim od tri clana trazi se broj strana 10-12, sto je u redu i minimum 10 referenci, a ovde imamo svega 6.
\odgovor{Dodate su neke reference i rad sada sadrži 10 referenci.}
\item Da li su slike i tabele funkcionalne i adekvatne?\\
Da.
\end{enumerate}

\section{Ocenite sebe}
Srednje upucen. Samo na kursevima na osnovnim studijama na Matematickom fakultetu. Nikakvo dodatno samostalo izucavanje na temu.
% Napišite koliko ste upuceni u oblast koju recenzirate: 
% a) ekspert u datoj oblasti
% b) veoma upuceni u oblast
% c) srednje upuceni
% d) malo upuceni 
% e) skoro neupuceni
% f) potpuno neupuceni
% Obrazložite svoju odluku


\chapter{Recenzent \odgovor{--- ocena:} }


\section{O čemu rad govori?}
% Напишете један кратак пасус у којим ћете својим речима препричати суштину рада (и тиме показати да сте рад пажљиво прочитали и разумели). Обим од 200 до 400 карактера.

Rad ukratko govori o formalnoj semantici programskih jezika, navodeći definicije različitih grupa formalizama (naime, operacione, denotacione i aksiomatske semantike), i 
primere upotrebe na jednostavnim programskim konstruktima.

\section{Krupne primedbe i sugestije}
% Напишете своја запажања и конструктивне идеје шта у раду недостаје и шта би требало да се промени-измени-дода-одузме да би рад био квалитетнији.

Zaključak i poslednji pasus prethodnog odeljka su identični. Jasno je da je u pitanju lapsus.\\
\odgovor{Izbrisan je dupliran pasus.}\\
\section{Sitne primedbe}
% Напишете своја запажања на тему штампарских-стилских-језичких грешки
Denotaciono stablo u primeru 3.1 je previše sitno.\\
\odgovor{Povećan je font u denotacionom stablu.}\\
Na strani 3 nakon navođenja strukturne semantike, postoji nekakva tačka.\\
\odgovor{Obrisana je tačka, a dodata referenca.}\\
Neuspešno ubacivanje referenci na stranama 4 i 5.\\
\odgovor{Sve greške sa referencama su ispravljene u celom radu.}\\
Na četvrtoj strani pri opisima konfiguracija pretpostavljam da su autori za završno stanje hteli da napišu s'.\\
\odgovor{Na tom mestu je ispravljeno završno stanje iz \textit{s} u \textit{s'}.}\\
Koristi se izraz "operacijska semantika" u zaključku nasuprot izrazu "operaciona semantika"  koji se koristio tokom ostatka rada. \\
\odgovor{Izraz "operacijska semantika" je ispravljen u "operaciona semantika" u zaključku.}\\
Čini mi se da je češći prevod "parcijalna korektnost" u domaćoj literaturi za {\em partial corectness}. \\
\odgovor{Ispravljeno.}\\
\section{Provera sadržajnosti i forme seminarskog rada}
% Oдговорите на следећа питања --- уз сваки одговор дати и образложење

\begin{enumerate}
\item Da li rad dobro odgovara na zadatu temu?\\
Da. Daje kratak opis semantika i mislim da je koristan za osobe koje se tek upoznaju sa materijom.
\item Da li je nešto važno propušteno?\\
Nije bilo elaboracije na temu koja semantika odgovara funkcionalnim a koja imperativnim jezicima. 
\odgovor{U poglavlju 2.1 i u poglavlju 2.2 su dodati pasusi koji govore da se za imperativne jezike koristi operaciona, a za funkcionalne jezike denotaciona semantika.}
\item Da li ima suštinskih grešaka i propusta?\\
\item Da li je naslov rada dobro izabran?\\
Možda bi bio adekvatniji neki naslov koji ističe da je u pitanju kratak pregled.\\
\odgovor{Naslov je prepravljen u Osnove formalne semantike programskih jezika}
\item Da li sažetak sadrži prave podatke o radu?\\
Da.
\item Da li je rad lak-težak za čitanje?\\
Lak je. Ne zalazi previše duboko u tematiku i delovi rada su nezavisne celine koje se mogu pojedinačno tumačiti.
\item Da li je za razumevanje teksta potrebno predznanje i u kolikoj meri?\\
Neophodno je baratanje uobičajenom programerskom terminologijom.
\item Da li je u radu navedena odgovarajuća literatura?\\
Da.
\item Da li su u radu reference korektno navedene?\\
Da. Detaljnom inspekcijom utvrđeni su konkretni delovi teksta na koje rad referiše.
\item Da li je struktura rada adekvatna?\\
Da.
\item Da li rad sadrži sve elemente propisane uslovom seminarskog rada (slike, tabele, broj strana...)?\\
Minimalan broj referenci je 8, rad sadrži 6.\\
\odgovor{Dodate su neke reference i rad sada sadrži 10 referenci.}
\item Da li su slike i tabele funkcionalne i adekvatne?\\
Da.
\end{enumerate}

\section{Ocenite sebe}
% Napišite koliko ste upućeni u oblast koju recenzirate: 
% a) ekspert u datoj oblasti
% b) veoma upućeni u oblast
% c) srednje upućeni
% d) malo upućeni 
% e) skoro neupućeni
% f) potpuno neupućeni
% Obrazložite svoju odluku

Izjašnjavam se kao malo upućen u oblast. Jasno mi je šta je semantika i značaj formalne dodele semantike programima, upoznat sam sa pojedinim semantikama, i
svi termini u seminarskom su mi bili makar poznati.  


\chapter{Recenzent \odgovor{--- ocena:} }


\section{O čemu rad govori?}
% Напишете један кратак пасус у којим ћете својим речима препричати суштину рада (и тиме показати да сте рад пажљиво прочитали и разумели). Обим од 200 до 400 карактера.
Seminarski rad opisuje semantiku programskih jezika, njeno značenje i ulogu. Jasno je opisana i naznačena važnost opisa semantike formalnim jezikom. Takođe je dat okviran opis operacione, denotacione i aksiomatske semantike. 

\section{Krupne primedbe i sugestije}
% Напишете своја запажања и конструктивне идеје шта у раду недостаје и шта би требало да се промени-измени-дода-одузме да би рад био квалитетнији.

Rad je prilično dobro odrađen i kao takav je veoma lako čitljiv. Krupnijih zamerki zaista nemam.

\section{Sitne primedbe}
% Напишете своја запажања на тему штампарских-стилских-језичких грешки
Kod navođenja C-ovskih izraza, u tekstu uvoda, preglednije bi bilo staviti ih u novi red i tako odvojiti od ostatka teksta. \\
\odgovor{Autori misle da je nepotrebno izdvajati ove izraze u poseban red jer su prikazani samo kao primeri nekih izraza, nigde se kasnije ne objašnjavaju. Njihovo izdvajanje u novi red bi dovelo do teže čitljivosti rečenice.}\\

U tekstu pod naslovom 2 \emph{Formalna semantika}, poslednja rečenica (\emph{'Na slici 1 dat je prikaz kompajlera'}), preciznije je reći da je dat \emph{'prikaz strukture kompajlera'}.\\
\odgovor{Ispravljeno je u ,,Na slici 1 dat je prikaz strukture kompilatora.''}\\

U tekstu pod naslovom 3 \emph{Operaciona semantika}, prva rečenica nakon navođenja vrsti istih, napisano je \emph{'apstraknih'} umesto \emph{'apsrtaktnih'}.\\
\odgovor{Reč ,,apstraknih'' je ispravljena u ,,apstraktnih''}.\\
U poslednjoj rečenici na strani 3, umesto \emph{'masinu'} trebalo bi da piše \emph{'mašinu'}, takođe u istoj rečenici \emph{'c'} treba zameniti sa \emph{'ć'}.
 U istom pasusu takođe ima greški sa nekorišćenjem slova srpske abecede.\\
\odgovor{Sva ogoljena latinična slova u tekstu su ispravljena.}\\
 U istom pasusu, kod navođenja algoritma zapisom instrukcija apstraktne mašine nije jasan simbol koji je naveden u \emph{if} naredbi. \\
\odgovor{Nejasan simbol je zamenjen simbolom $>$.}\\
 Takođe u narednom pasusu napisano je \emph{'apstraknu'} umesto \emph{'apstraktnu'}. \\
\odgovor{Prepravljeno.}\\
Referenca na u narednom pasusu nije lepo navedena, a takođe i na narednoj strani.\\
\odgovor{Sve greške se referencama su ispravljene.}\\ 
Kod navođenja značenja engleskih reči poželjno bi bilo koristiti '\emph{emph}' obeleživač.\\
\odgovor{Da li je ovo zaista neko pravilo? Autori se do sada nisu susreli sa ovim.}\\

U prvom pasusu u tekstu pod naslovom 5 \emph{Aksiomatska semantika}, u petoj rečenici, napisano je \emph{'logicki'} umesto \emph{'logički'}.
 U narednoj rečenici i definiciji 5.1 greška sa istim slovom.\\
\odgovor{Sva ogoljena latinična slova u tekstu su ispravljena.}\\
  Drugi pasus na strani 9 u reči \emph{'sematike'} je izostavljeno slovo \emph{'n'}.\\
   \odgovor{Prepravljeno.}\\

\section{Provera sadržajnosti i forme seminarskog rada}
% Oдговорите на следећа питања --- уз сваки одговор дати и образложење

\begin{enumerate}
\item Da li rad dobro odgovara na zadatu temu?\\
Moje zapažanje je da rad prilično dobro opisuje zadatu temu, jer je kroz primere slikovito opisana problematika. 
\item Da li je nešto važno propušteno?\\
Moj utisak je da nije, jer su autori i na samom početku rekli da je ovo samo površan opis podoblasti zadate teme.
\item Da li ima suštinskih grešaka i propusta?\\
Nije bilo takvih.
\item Da li je naslov rada dobro izabran?\\
Jeste, u potpunosti.
\item Da li sažetak sadrži prave podatke o radu?\\
Apstrakt(sažetak) je jasno predstavio ciljeve i svrhe rada, takođe i sadrži ključne reči.
\item Da li je rad lak-težak za čitanje?\\
Rad je prilično lak i razumljiv za čitaoce.
\item Da li je za razumevanje teksta potrebno predznanje i u kolikoj meri?\\
Predznanje iz računarskih tehnologija je poželjno, ali ne i neophodno. 
\item Da li je u radu navedena odgovarajuća literatura?\\
U potpunosti.
\item Da li su u radu reference korektno navedene?\\
U potpunosti.
\item Da li je struktura rada adekvatna?\\
Sadrži sve neophodne strukturne elemente, koji su jasno koncipirani.
\item Da li rad sadrži sve elemente propisane uslovom seminarskog rada (slike, tabele, broj strana...)?\\
U potpunosti.
\item Da li su slike i tabele funkcionalne i adekvatne?\\
Slike i tabele su na pravom mestu i ogovaraju sadržaju.
\end{enumerate}

\section{Ocenite sebe}
% Napišite koliko ste upućeni u oblast koju recenzirate: 
% a) ekspert u datoj oblasti
% b) veoma upućeni u oblast
% c) srednje upućeni
% d) malo upućeni 
% e) skoro neupućeni
% f) potpuno neupućeni
% Obrazložite svoju odluku
Veoma upućen. Bavim se programiranjem na fakultetu i van njega, i semantika kao oblast istraživanja mi je prilično poznata.


\chapter{Dodatne izmene}
%Ovde navedite ukoliko ima izmena koje ste uradili a koje vam recenzenti nisu tražili. 
\textbf{Apstrakt: }
\begin{enumerate}
\item u ključnim rečima zamenjeno veliko \textit{S} u Semantika programskih jezika u malo \textit{s}.
\end{enumerate}
\textbf{Formalna semantika:}
\begin{enumerate}
\item izraz \texttt{$ while(x > 0)$  $\lbrace x--;\rbrace$} je izdvojen u novi red radi čitljivosti
\end{enumerate}
\textbf{Zaključak:}
\begin{enumerate}
\item Ceo zaključak je izmenjen jer je jako ličio na uvod.
\end{enumerate}
\end{document}
