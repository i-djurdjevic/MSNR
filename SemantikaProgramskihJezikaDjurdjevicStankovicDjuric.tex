% !TEX encoding = UTF-8 Unicode

\documentclass[a4paper]{article}

\usepackage{color}
\usepackage{url}
\usepackage[utf8]{inputenc} % make weird characters work
\usepackage{graphicx}

%Milica: meni ne radi ovde serbian, samo croatian :(
\usepackage[english,croatian]{babel}


\usepackage[unicode]{hyperref}
\hypersetup{colorlinks,citecolor=green,filecolor=green,linkcolor=blue,urlcolor=blue}

%\newtheorem{primer}{Пример}[section] %ćirilični primer
\newtheorem{primer}{Primer}[section]
\newtheorem{definicija}{Definicija}[section]

\begin{document}

\title{Naslov seminarskog rada\\ \small{Seminarski rad u okviru kursa\\Metodologija stručnog i naučnog rada\\ Matematički fakultet}}

\author{Isidora Đurđević, Ana Stanković, Milica Đurić\\ kontakt email prvog, drugog (trećeg) autora}
\date{5.~april 2017.}
\maketitle

\abstract{
U ovom tekstu je ukratko prikazana osnovna forma seminarskog rada. Obratite pažnju da je pored ove .pdf datoteke, u prilogu i odgovarajuća .tex datoteka, kao i .bib datoteka korišćena za generisanje literature. Na prvoj strani seminarskog rada su naslov, apstrakt i sadržaj, i to sve mora da stane na prvu stranu! Kako bi Vaš seminarski zadovoljio standarde i očekivanja, koristite uputstva i materijale sa predavanja na temu pisanja seminarskih radova. Ovo je samo šablon koji se odnosi na fizički izgled seminarskog rada (šablon koji \emph{morate} da ispoštujete!) kao i par tehničkih pomoćnih uputstava. Molim Vas da kada budete predavali seminarski rad, imenujete datoteke tako da sadrže temu seminarskog rada, kao i imena i prezimena članova grupe (ili samo temu i prezimena, ukoliko je sa imenima predugačko). Predaja seminarskih radova biće isključivo preko web forme, a NE slanjem mejla.

\tableofcontents

\newpage

\section{Uvod}
\label{sec:uvod}
Ovde pišem uvod
\begin{primer}
Problem zaustavljanja (eng.~{\em halting problem}) je neodlučiv \cite{haltingproblem}.
\end{primer}


\section{Operaciona semantika}
\label{sec:opsem}


\section{Denotaciona semantika}
\label{sec:densem}
Nastala 1960. godina od strane Christopher Strachey-a i njegove istraživačke grupe na Oxford-u \cite{slonneger1995book}, \textit{denotaciona semantika} predstavlja jednu vrstu reakcije na operacionu semantiku za koju se smatra da sadrži puno informacija. Naziv je dobila po engleskoj reči označiti (eng.~{\em denote}) jer pridružuje značenja sintaksnim definicijama jezika. Alternativno, može se nazivati i \textit{matematička semantika} zbog njene okrenutosti matematičkim formalizmima pri definisanju ove formalne semantike. Jedan način definisanja denotacione semantike je dat u sledećoj definiciji.
\begin{definicija}
Pristup formalizaciji semantike konstruisanjem matematičkih objekata koji
opisuju značenje jezika naziva se \textbf{denotaciona semantika} \cite{milena}.
\end{definicija}

\section{Aksiomatska semantika}
\label{sec:akssem}
Za nastanak i razvoj \textit{aksiomatske semantike}  su zaslužni pre svega Floyd, Hoar i Dijkstra \cite{kuncak}.
Aksiomatska semantika razvija metode za proveru korektnosti programa. Zasniva se na matematickoj logici. Za svaku kontrolnu strukturu i komandu se definišu logicki izrazi. Ovi izrazi se nazivaju tvrđenja (eng.~{\em  assertions}) i u njima se zadaju ogranicenja za promenljive koja se javljaju u tim kontrolnim strukturama i komandama.\\
Tvrđenja su data u obliku  Horovih trojki:
\{P\}c\{Q\} \\
\begin{definicija}
\textbf{Horova trojka \{P\}C\{Q\}} opisuje kako izvršavanje dela koda menja stanje izracunavanja ako je ispunjen preduslov (eng.~{\em  precondition}) \{P\}, izvršavanje komande C vodi do postuslova (eng.~{\em  postcondition}) \{Q\} \cite{milena} .  \\

\end{definicija}

Preduslov je logički izraz u kome se definišu ograničenja promenljivih pre izvršavanja komande, a postuslov definiše ograničenja promenljivih posle izvršavanja komande.\\
Horove trojke se drugačije nazivaju i \textit{delimična ispravnost specifikacije } (eng.~{\em  partial correctness specification}) . \\
Ali one ne mogu da osiguraju da će se program završiti pa se zbog toga i nazivaju “delimičnim”. \\
Pored delimične ispravnosti specifikacije, imamo i \textit {potpunu ispravnost naredbe } (eng.~{\em  total correctness statements}) koja osigurava da će se program završiti dok god preduslov važi. \\
Preduslovi i postuslovi mogu se smatrati interfejsom ili ugovorom između programa i njegovih klijenata. Oni pomažu korisnicima da razumeju šta program treba da proizvede bez potrebe da shvati kako se program izvršava. Tipično, programeri ih pišu kao komentari za funkcije i
funkcionišu kao dokumentacija i olakšavaju održavanje programa. Takve specifikacije su posebno
korisne za bibliotečke funkcije za koje izvorni kod često nije dostupan korisnicima \cite{adrian}. 

Način funkcionisanja ove semantike možemo prikazati u primeru koji sledi (primer je preuzet iz knjige \cite{nielson} ). \\
\begin{primer}
\{ x=n \} \\
 y := 1; 
 while neq{(x=1)} do (y := x*y; x := x-1)\\
 \{ y=n! and n veće od 0 \} \\
 
n je u primeru specijalna promenljiva koja se naziva logička promenljiva i koja, za razliku od programskih promenljivih, ne sme se pojaviti ni u jednoj naredbi koja se izvršava u programu i njena vrednost će uvek biti ista. Njena uloga je da pamte inicijalne vrednosti programskih promenljivih . \\
 Zapisali smo  preduslov da promenljive n ima jednaku vrednost kao i x u početnom stanju (tj. nego što program sa faktorijalom krene da se izvršava). S obzirom da program neće promeniti vrednost promenljive n, postuslov y=n! će izraziti da konačna vrednost y  će biti jednaka faktorijalu početne vrednosti promenljive x, kada se izvršavanje programa završi. \\
  
\end{primer}




\section{Zaključak}
\label{sec:zakljucak}

Ovde pišem zaključak. 



\addcontentsline{toc}{section}{Literatura}
\appendix
\bibliography{seminarski} 
\bibliographystyle{plain}

\appendix

\end{document}
