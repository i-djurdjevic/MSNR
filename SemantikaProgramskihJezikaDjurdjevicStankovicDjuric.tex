% !TEX encoding = UTF-8 Unicode

\documentclass[a4paper]{article}

\usepackage{color}
\usepackage{url}
\usepackage[utf8]{inputenc} % make weird characters work
\usepackage{graphicx}

%Milica: meni ne radi ovde serbian, samo croatian :(
\usepackage[english,croatian]{babel}


\usepackage[unicode]{hyperref}
\hypersetup{colorlinks,citecolor=green,filecolor=green,linkcolor=blue,urlcolor=blue}

%\newtheorem{primer}{Пример}[section] %ćirilični primer
\newtheorem{primer}{Primer}[section]
\newtheorem{definicija}{Definicija}[section]

\begin{document}

\title{Naslov seminarskog rada\\ \small{Seminarski rad u okviru kursa\\Metodologija stručnog i naučnog rada\\ Matematički fakultet}}

\author{Isidora Đurđević, Ana Stanković, Milica Đurić\\ kontakt email prvog, drugog (trećeg) autora}
\date{5.~april 2017.}
\maketitle

\abstract{
U ovom tekstu je ukratko prikazana osnovna forma seminarskog rada. Obratite pažnju da je pored ove .pdf datoteke, u prilogu i odgovarajuća .tex datoteka, kao i .bib datoteka korišćena za generisanje literature. Na prvoj strani seminarskog rada su naslov, apstrakt i sadržaj, i to sve mora da stane na prvu stranu! Kako bi Vaš seminarski zadovoljio standarde i očekivanja, koristite uputstva i materijale sa predavanja na temu pisanja seminarskih radova. Ovo je samo šablon koji se odnosi na fizički izgled seminarskog rada (šablon koji \emph{morate} da ispoštujete!) kao i par tehničkih pomoćnih uputstava. Molim Vas da kada budete predavali seminarski rad, imenujete datoteke tako da sadrže temu seminarskog rada, kao i imena i prezimena članova grupe (ili samo temu i prezimena, ukoliko je sa imenima predugačko). Predaja seminarskih radova biće isključivo preko web forme, a NE slanjem mejla.

\tableofcontents

\newpage

\section{Uvod}
\label{sec:uvod}
Ovde pišem uvod
\begin{primer}
Problem zaustavljanja (eng.~{\em halting problem}) je neodlučiv \cite{haltingproblem}.
\end{primer}


\section{Operaciona semantika}
\label{sec:opsem}


\section{Denotaciona semantika}
\label{sec:densem}
Nastala 1960. godina od strane Christopher Strachey-a i njegove istraživačke grupe na Oxford-u \cite{slonneger1995book}, \textit{denotaciona semantika} predstavlja jednu vrstu reakcije na operacionu semantiku za koju se smatra da sadrži puno informacija. Naziv je dobila po engleskoj reči označiti (eng.~{\em denote}) jer pridružuje značenja sintaksnim definicijama jezika. Alternativno, može se nazivati i \textit{matematička semantika} zbog njene okrenutosti matematičkim formalizmima pri definisanju ove formalne semantike. Jedan način definisanja denotacione semantike je dat u sledećoj definiciji.
\begin{definicija}
Pristup formalizaciji semantike konstruisanjem matematičkih objekata koji
opisuju značenje jezika naziva se \textbf{denotaciona semantika} \cite{milena}.
\end{definicija}

\section{Aksiomatska semantika}
\label{sec:akssem}

\section{Zaključak}
\label{sec:zakljucak}

Ovde pišem zaključak. 



\addcontentsline{toc}{section}{Literatura}
\appendix
\bibliography{seminarski} 
\bibliographystyle{plain}

\appendix

\end{document}
