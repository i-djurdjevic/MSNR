\documentclass{beamer}
\usetheme{Singapore}
\usecolortheme{dolphin}
\usepackage[utf8x,utf8]{inputenc} % make weird characters work
\usepackage[croatian]{babel}
\usepackage{tcolorbox}
\usepackage{color}
\setbeamertemplate{footline}[frame number]
\title{Osnove formalne semantike programskih jezika}
\subtitle{Seminarski rad u okviru kursa \\ Metodologija stručnog i naučnog rada}
\institute[Matematički fakultet]{isidoradjurdjevic.100@gmail.com,\\anastankovic167@gmail.com,\\ mdjuric55@gmail.com
\medskip 
\medskip
\\Matematički fakultet, Beograd
}
\author{Isidora Đurđević, Ana Stanković, Milica Đurić}
\date{4.~maj 2017.}

\begin{document}

\maketitle

\begin{frame}{Pregled}
  \tableofcontents
\end{frame}


\section{Semantika programskih jezika}

\subsection{Odnos sintakse i semantike}
\begin{frame}{Odnos sintakse i semantike}
  \begin{itemize}
  \item Kao i prirodni jezici, programski jezici se bave izučavanjem sintakse i semantike.
  \item Sintaksa se bavi strukturom programskog jezika.
  \item {\color{magenta} Semantika se bavi značenjem programskog jezika.}
  \item Dve vrste semantike:
  \begin{itemize}
  \item {\color{blue} neformalna }
  \item {\color{blue} formalna }
  \end{itemize}
  
  \end{itemize}
 
\end{frame}


\section{Formalna semantika}
\subsection{Motivacija}
\begin{frame}{Formalna semantika programskih jezika} 
 \begin{itemize}
 \item Motivacija:
 \begin{itemize}
 \item {\color{magenta} umanjuje nepreciznosti} koje se javljaju u neformalnom definisanju semantike
 \item daje neki vid {\color{magenta} opisa logike programskog jezika } za koji je definisana
 \item čini osnovu formalne verifikacije programa {\color{magenta} (deo je kompilatora)}
  \end{itemize}
 \item Pristupi definisanja:
 \begin{itemize}
 \item \color{blue}operaciona semantika
 \item denotaciona semantika
 \item aksiomatska semantika
 \end{itemize}
 \end{itemize}
\end{frame}



\subsection{Operaciona semantika}
\begin{frame}{Operaciona semantika}
 	\begin{itemize}
  		\item način davanja značenja programskim jezicima kroz matematičku reprezentaciju
		\item {\color{magenta}opisuje kako se stanje programa menja tokom izvršavanja programa}
		\item niz računarskih koraka predstavlja značenje programa
		\item klasifikovana je u dve kategorije: 
			\begin{itemize}
				\item {\color{blue} prirodna semantika}
				\item {\color{blue} strukturna operaciona semantika}
			\end{itemize}
		\item ponašanje programa se definiše tranzicionim sistemom
 	\end{itemize}
\end{frame}

\begin{frame}{Prirodna operaciona semantika}
	\begin{itemize}
 	\item ,,veliki koraci"
 	\item {\color{magenta} opisuje krajnje rezultate izvršavanja}
 	\item predstavlja apstrakciju
 	\item opisuje vezu između početnog i završnog stanja izvršavanja programa
 	\item često jednostavnija za upotrebu
 	\end{itemize}
\end{frame}

\begin{frame}{Strukturna operaciona semantika}
\begin{itemize}
	\item ,,mali koraci"
	\item  {\color{magenta} opisuje kako se svaki korak programa izvršava}
	\item koristi se za obimne analize i složene slučajeve
	\item semantika pokazivača, višenitne obrade, goto naredbe...
	\item veća kontrola nad detaljima
	\end{itemize}
\end{frame}
\subsection{Denotaciona semantika}
\begin{frame}{Denotaciona semantika}
  \begin{itemize}
  \item nastala 1960. godina
  \item drugi naziv je matematička semantika
  \item {\color{magenta} predstavlja pristup formalizaciji semantike konstruisanjem matematičkih objekata }
  \item bitan samo odnos između početnog i završnog stanja, a ne koraci izvršavanja programa
   \item bitna osobina -{\color{magenta} kompozitivnost }
   \begin{itemize}
   \item semantika jedne programske celine definisana je preko semantike njenih poddelova
   \item izraz $(15 + 3) * (2 + 2)$ ima isto značenje kao $18 * (2 + 2)$
   \end{itemize}
   \item najbolja za definisanje semantike funkcionalnih jezika
   \end{itemize}
\end{frame}

\begin{frame}{Semantika aritmetičkih izraza}

\begin{tcolorbox}
   \textbf{Sintaksni domeni i pravila:}
\begin{itemize}

\item$B: Broj $ , $C: Cifra $ , $I: Izraz $\\
\item$ Broj ::== Cifra | Broj Cifra $, $ Cifra ::== 0 | 1 | 2 | 3 | 4 | 5 | 6 | 7 | 8 | 9 $, $ Izraz ::== Broj | Izraz+Izraz $
\end{itemize}
   \textbf{Semantički domeni}
\begin{itemize}
\item $N={0,1,2,3,....} $\\
\end{itemize}
\end{tcolorbox}

\end{frame}
\begin{frame}
\begin{tcolorbox}
    \textbf{Funkcije značenja}
\begin{itemize}
\item $povbn: B \rightarrow N $ , $povcn: C \rightarrow N $ , $sem: I \rightarrow N $   \\

\item $plus: N \times N \rightarrow N $ , $pom: N \times N \rightarrow N $ \\

\item $ povcn[[0]] = 0,... ,povcn[[9]] = 9 $\\

\item $ povbn[[C]] = povcn[[C]] $\\
$ povbn[[B C]]\mbox{ = plus(pom(10,povezibn[[B]]),povbn[[C]])} $

\item $ sem[[B]] = povbn[[B]] $, $ sem[[I1 + I2]] = plus(sem[[I1]],sem[[I2]]) $
\end{itemize}
 
\end{tcolorbox}

\begin{itemize}
\item Koja je semantička vrednost izraza $2 + 11$?
\end{itemize}
\end{frame}
\subsection{Aksiomatska semantika}
\begin{frame}{Aksiomatska semantika}
  \begin{itemize}
  \item za njen nastanak i razvoj zaslužni Flojd, Hor i Dijkstra
  \item {\color{magenta}zasniva se na matematičkoj logici}
  \item razvija metode za proveru korektnosti programa
   \item tvrđenja su data u obliku Horovih trojki \{P\}C\{Q\}
   \item{\color{magenta} preduslov} je logički izraz u kome se definišu ograničenja promenljivih pre izvršavanja komande
   \item {\color{magenta} postuslov} je logički izraz u kome se definišu ograničenja promenljivih posle izvršavanja komande
  \end{itemize}
\end{frame}

\begin{frame}{Aksiomatska semantika}
 \begin{itemize}
  \item postoji parcijalna ispravnost specifikacije i potpuna ispravnost naredbe
  \item uloga aksiomatske semantike
  \begin{itemize}
  \item dokazivanje korektnosti programa i algoritama
   \item proširena statička provera (npr. provera granice niza)
   \item dokumentacija programa i interfejsa
    \end{itemize} 
    \item primer sa faktorijalom
  \begin{tcolorbox}
  \begin{center}


\texttt{\{ x = n \}}   \\
\texttt{ y:=1;} 
 \texttt{while $ \neg(x=1) $   do  (y:=x*y; x:=x-1) }\\
\texttt{ \{ y=n! and  n>0 \}  } \\
\end{center}
\end{tcolorbox}
  \end{itemize} 
\end{frame}
\section{Dalje istraživanje}
\begin{frame}{Dalje istraživanje}
  \begin{itemize}
  \item obuhvaćene su samo osnove formalne semantike
  \item dalje istraživanje bi obuhvatalo:
  \begin{itemize}
  	 \item semantičko definisanje kompleksnih programskih fragmenata koji uključuju rekurziju ili goto naredbu, i slično
  	 \item ulogu formalne semantike pri konstruisanju kompilatora
  	 \item pregled formalne semantike nekog programskog jezika, npr. Hasekell
  \end{itemize}
  

  \end{itemize}
\end{frame}
\section{Literatura}
\begin{frame}{Literatura}
  \begin{itemize}
  \item Flemming Neilson, {\color{blue} Hanne Riis Nielson, Semantics with applications},
John Willey and Sons, 1999.
  \item K. Slonneger and B. Kurtz, {\color{blue}Formal Syntax and Semantics of Programming Languages}, Addison-Wesley Publishing Company, United
States of America, 1995.
  \item M. Vujošević Janičić, {\color{blue}Dizajn programskih jezika, Osnovna svojstva programskih jezika}, Beograd, 2016.
  \item B. Evan Chang, {\color{blue}Introduction to Axiomatic Semantics}, United States of America, 2009.
  \end{itemize}
\end{frame}
\end{document}