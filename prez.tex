\documentclass{beamer}
\usetheme{Singapore}
\usecolortheme{dolphin}
\usepackage[utf8x,utf8]{inputenc} % make weird characters work
\usepackage[croatian]{babel}
\usepackage{tcolorbox}
\setbeamertemplate{footline}[frame number]
\title{Osnove formalne semantike programskih jezika}
\subtitle{Seminarski rad u okviru kursa \\ Metodologija stručnog i naučnog rada}
\institute[Matematički fakultet]{isidoradjurdjevic.100@gmail.com,\\anastankovic167@gmail.com,\\ mdjuric55@gmail.com
\medskip 
\medskip
\\Matematički fakultet, Beograd
}
\author{Isidora Đurđević, Ana Stanković, Milica Đurić}
\date{4.~maj 2017.}

\begin{document}

\maketitle

\begin{frame}{Pregled}
  \tableofcontents
\end{frame}


\section{Semantika programskih jezika}

\subsection{Odnos sintakse i semantike}
\begin{frame}{Odnos sintakse i semantike}
  \begin{itemize}
  \item Kao i prirodni jezici, programski jezici se bave izučavanjem sintakse i semantike.
  \item Sintaksa se bavi strukturom programskog jezika.
  \item Semantika se bavi značenjem programskog jezika.
  \begin{itemize}
  \item neformalna semantika
  \item formalna semantika
  \end{itemize}
  
  \end{itemize}
 
\end{frame}


\section{Formalna semantika}
\subsection{Motivacija}
\begin{frame}{Formalna semantika programskih jezika} 
 \begin{itemize}
 \item Motivacija:
 \begin{itemize}
 \item umanjuje nepreciznosti koje se javljaju u neformalnom definisanju semantike
 \item daje neki vid opisa logike programskog jezika za koji je definisana
 \item čini osnovu formalne verifikacije programa (deo je kompilatora)
  \end{itemize}
 \item Pristupi definisanja:
 \begin{itemize}
 \item operaciona semantika
 \item denotaciona semantika
 \item aksiomatska semantika
 \end{itemize}
 \end{itemize}
\end{frame}



\subsection{Operaciona semantika}
\begin{frame}{Operaciona semantika}
 	\begin{itemize}
  		\item nacin davanja znacenja programskim jezicima kroz matematicku reprezentaciju
		\item opisuje kako se stanje programa menja tokom izvrsavanja programa
		\item niz racunarskih koraka predstavlja znacenje programa
		\item klasifikovana je u dve kategorije: 
			\begin{itemize}
				\item prirodna semantika
				\item strukturna operaciona semantika
			\end{itemize}
		\item ponasanje programa se definise tranzicionim sistemom
 	\end{itemize}
\end{frame}

\begin{frame}{Prirodna operaciona semantika}
 	\item ,,veliki koraci"
 	\item opisuje krajnje rezultate izvrsavanja
 	\item predstavlja apstrakciju
 	\item opisuje vezu izmedju pocetnog i zavrsnog stanja izvrsavanja programa
 	\item cesto jednostavnija za upotrebu
\end{frame}

\begin{frame}{Strukturna operaciona semantika}
	\item ,,mali koraci"
	\item  opisuje kako se svaki korak programa izvrsava
	\item koristi se za obimne analize i slozene slucajeve
	\item semantika pokazivaca, visenitne obrade, goto naredbe...
	\item veca kontrola nad detaljima
\end{frame}
\subsection{Denotaciona semantika}
\begin{frame}{Denotaciona semantika}
  \begin{itemize}
  \item nastala 1960. godina
  \item drugi naziv je matematička semantika
  \item predstavlja pristup formalizaciji semantike konstruisanjem matematičkih objekata 
  \item bitan samo odnos između početnog i završnog stanja, a ne koraci izvršavanja programa
   \item bitna osobina - kompozitivnost
   \begin{itemize}
   \item semantika jedne programske celine definisana je preko semantike njenih poddelova
   \item izraz $15 + 3$ ima isto značenje kao $16 + 2$
   \end{itemize}
   \end{itemize}
\end{frame}

\begin{frame}{lsldjskdjks}
  
\end{frame}
\subsection{Aksiomatska semantika}
\begin{frame}{Aksiomatska semantika}
  \begin{itemize}
  \item za njen nastanak i razvoj zaslužni Flojd, Hor i Dijkstra
  \item zasniva se na matematičkoj logici
  \item razvija metode za proveru korektnosti programa
   \item tvrđenja su data u obliku Horovih trojki \{P\}C\{Q\}
   \item preduslov je logički izraz u kome se definišu ograničenja promenljivih pre izvršavanja komande
   \item postuslov je logički izraz u kome se definišu ograničenja promenljivih posle izvršavanja komande
  \end{itemize}
\end{frame}

\begin{frame}{Aksiomatska semantika}
 \begin{itemize}
  \item postoji parcijalna ispravnost specifikacije i potpuna ispravnost naredbe
  \item uloga aksiomatske semantike
  \begin{itemize}
  \item dokazivanje korektnosti programa i algoritama
   \item proširena statička provera (npr. provera granice niza)
   \item dokumentacija programa i interfejsa
    \end{itemize} 
    \item primer sa faktorijalom
  \begin{tcolorbox}
  \begin{center}


\texttt{\{ x = n \}}   \\
\texttt{ y:=1;} 
 \texttt{while $ \neg(x=1) $   do  (y:=x*y; x:=x-1) }\\
\texttt{ \{ y=n! and  n>0 \}  } \\
\end{center}
\end{tcolorbox}
  \end{itemize} 
\end{frame}
\section{Dalje istraživanje}
\begin{frame}{Dalje istraživanje}
  \begin{itemize}
  \item obuhvaćene su samo osnove formalne semantike
  \item dalje istraživanje bi obuhvatalo:
  \begin{itemize}
  	 \item semantičko definisanje kompleksnih programskih fragmenata koji uključuju rekurziju ili goto naredbu, i slično
  	 \item ulogu formalne semantike pri konstruisanju kompilatora
  	 \item pregled formalne semantike nekog programskog jezika, npr. Hasekell
  \end{itemize}
  

  \end{itemize}
\end{frame}
\section{Literatura}
\begin{frame}{Literatura}
  
\end{frame}
\end{document}